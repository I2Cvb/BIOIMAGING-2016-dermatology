\documentclass[a4paper,twoside]{article}

%% Latex documents that need direct input
\input{latex/filesystem/package.tex}        % contains the latex packages
%\title{Tackling the Curse of Data Imbalancing for Melanoma Classification}

\author{\authorname{First Author Name\sup{1}, Second Author Name\sup{2}}
\affiliation{\sup{1}Institute1,University1,Street1,Town1,Country1}
\affiliation{\sup{2}Institute2,University2,Street2,Town2,Country2}
\email{\{f\_author, s\_author\}@ips.xyz.edu, t\_author@dc.mu.edu}
}
             % contains the Title and Autor info
\input{latex/filesystem/fileSetup.tex}      % contains package and variables init.
\input{content/acronym_definition.tex}      % contains the acronims

\begin{document}

\title{Tackling the Curse of Data Imbalancing for Melanoma Classification}

\author{\authorname{First Author Name\sup{1}, Second Author Name\sup{1} and Third Author Name\sup{2}}
\affiliation{\sup{1}Institute of Problem Solving, XYZ University, My Street, MyTown, MyCountry}
\affiliation{\sup{2}Department of Computing, Main University, MySecondTown, MyCountry}
\email{\{f\_author, s\_author\}@ips.xyz.edu, t\_author@dc.mu.edu}    
}

\keywords{\uppercase{Imbalanced, Classification, Melanoma, Dermoscopy}}

\abstract{
\acresetall
Malignant melanoma is the most dangerous type of skin cancer, yet melanoma is the most treatable kind of cancer when diagnosed at an early stage.
In this regard, \ac{cad} systems based on machine learning have been developed to discern melanoma lesions from benign and dysplastic nevi in dermoscopic images.
Similar to a large range of real world applications encountered in machine learning, melanoma classification faces the challenge of imbalanced data. 
This article is devoted to analyze the impact of data balancing strategies at the training step.
Subsequently, an extensive comparison between \ac{os} and \ac{us}, in both feature and data space is performed revealing the fact that \ac{nm2} outperforms other methods achieving \ac{se} and \ac{sp} of 92.50 \% and 77.50 \%, respectively.
More generally, the reported results highlight that methods based on \ac{os} in data space and \ac{us} in feature space outperform the others.
}

\onecolumn \maketitle \normalsize \vfill
\acresetall  % reset the acronyms from the abstract

\include*{content/intro/intro}          % the file wihtout .tex
\include*{content/method/material-method}
\include*{content/method/method}
\include*{content/experiments-results/experiments-results}
\include*{content/conclusion/conclusion}


\bibliographystyle{apalike}
{\small
\bibliography{./content/literature_review}}

\include*{content/method/framework-result-ls}

\end{document}