\documentclass[a4paper,twoside]{article}

%% Latex documents that need direct input
\input{latex/filesystem/package.tex}        % contains the latex packages
%\title{Tackling the Curse of Data Imbalancing for Melanoma Classification}

\author{\authorname{First Author Name\sup{1}, Second Author Name\sup{2}}
\affiliation{\sup{1}Institute1,University1,Street1,Town1,Country1}
\affiliation{\sup{2}Institute2,University2,Street2,Town2,Country2}
\email{\{f\_author, s\_author\}@ips.xyz.edu, t\_author@dc.mu.edu}
}
             % contains the Title and Autor info
\input{latex/filesystem/fileSetup.tex}      % contains package and variables init.
\input{content/acronym_definition.tex}      % contains the acronims

\begin{document}

\title{Tackling the Problem of Data Imbalancing for Melanoma Classification}

\author{\authorname{Mojdeh Rastgoo\sup{1,2}, Guillaume Lemaitre\sup{1,2}, Joan Massich\sup{1}, Olivier Morel\sup{1}, Franck Marzani\sup{1}, Rafael Garcia\sup{2}, Fabrice Meriaudeau\sup{1}}
\affiliation{\sup{1}LE2I UMR6306, CNRS, Arts et M\'etiers, Universit\'e de Bourgogne Franche-Comt\'e, 12 rue de la Fonderie, 71200 Le Creusot, France}
\affiliation{\sup{2}ViCOROB, Universitat de Girona, Campus Montilivi, Edifici P4, 17071 Girona, Spain}
\email{\{mojdeh.rastgo-dastjerdi, guillaume.lemaitre, joan.massich, olivier.morel, franck.marzani and fmeriau\}@u-bourgogne.fr, rafael.garcia@udg.edu}    
}

\keywords{\uppercase{Imbalanced, Classification, Melanoma, Dermoscopy}}

\abstract{
\acresetall
Malignant melanoma is the most dangerous type of skin cancer, yet melanoma is the most treatable kind of cancer when diagnosed at an early stage.
In this regard, \ac{cad} systems based on machine learning have been developed to discern melanoma lesions from benign and dysplastic nevi in dermoscopic images.
Similar to a large range of real world applications encountered in machine learning, melanoma classification faces the challenge of imbalanced data, where the percentage of melanoma cases in comparison with benign and dysplastic cases is far less.
This article analyzes the impact of data balancing strategies at the training step.
Subsequently, \ac{os} and \ac{us} are extensively compared in both feature and data space, revealing that \ac{nm2} outperform other methods achieving \ac{se} and \ac{sp} of 91.2\% and 81.7\%, respectively.
More generally, the reported results highlight that methods based on \ac{us} or combination of \ac{os} and \ac{us} in feature space outperform the others.
}

\onecolumn \maketitle \normalsize \vfill
\acresetall  % reset the acronyms from the abstract

\include*{content/intro/intro}          % the file wihtout .tex
\include*{content/method/material-method}
\include*{content/method/method}
\include*{content/experiments-results/experiments-results}
\include*{content/conclusion/conclusion}


\bibliographystyle{apalike}
{\small
\bibliography{./content/literature_review}}



\end{document}