% include the figures path relative to the master file
\graphicspath{ {./content/Experiments-results/figures/} }

\section{\uppercase{Experimental Results}}
\label{sec:exp-res} 

\noindent The classification results are reported in Table~\ref{tab:tab1} using the aforementioned features, the \ac{rf} classifier, and the different imbalancing techniques presented in Sect.\,\ref{sec:met} and Sect.\,\ref{sec:clas-val}. 

Table~\ref{tab:tab1} can be divided into three main parts representing the results using imbalance data (IB), the balancing in the data space \ac{os} and the balancing in the feature space.
These strategies are separated by a double horizontal line.
The strategies performed in the feature space are subdivided into either \ac{os} or \ac{us} or a combination of \ac{os} follow by \ac{us} (see horizontal dashed line in Table~\ref{tab:tab1}).

In this table, based on the previously defined cost function, the best performance for each feature set are highlighted in the shaded cells.
Table~\ref{tab:tab2} shows the obtained cost value for each configuration. 
Strategies with low cost function are synonymous with a better \ac{se} and \ac{sp} trade-off.

The obtained results indicate that balancing techniques are essential and improve the classification performance. 
% Certainly, some techniques are more subjects to the feature representations than others. 
%However, the improvements in comparison to imbalanced classification is evident. 
For this case study the \ac{us} techniques and their combination with \ac{os} techniques outperform the \ac{os} techniques. 
Due to the characteristics similarities of melanoma and dysplastic lesions, it is expected to have correlated feature space among melanoma and dysplastic lesions. 
Subsequently, the miss-leading samples could be removed using \ac{us} and lead to better performance.
Specifically to our purpose, \ac{nm2} algorithm with the combination of all the features ($T_{1,2} C_{1,2}$) and combination of Gabor and color features ($T_{2}, C_{1,2}$), with the lowest costs function, achieve the highest \ac{se} and \ac{sp} of 92.50 \% and 78.33 \%, and 92.50 \% and 77.50 \%, respectively.  
Focusing only on \ac{os} techniques, \ac{os} in data space outperforms the techniques performing in feature space.

Comparing the color features, opponent color angle and hue histogram feature descriptor, $C_{2}$, has a better performance than well-used color statistics, $C_{1}$. 
In texture domain, Gabor descriptor, $T_{2}$, outperforms CLBP features, $T_{1}$.
Finally, the combination of color and texture features outperforms any other feature combination. 

%is the algorithm maximizing the sensitivity and in overall, \ac{nm} algorithms perform the best on our dataset. 
%However, \ac{ncr} algorithm (see results highlighted in blue in Table~\ref{tab:tab1}) achieves the best performance, considering a trade-off between \ac{se} and \ac{sp}. 
% Table~\ref{tab:tab1} show the obtained results in terms of 

% Some stuff that emac's colegues use
%%% Local Variables: 
%%% mode: latex
%%% TeX-master: "../../master"
%%% End: 

