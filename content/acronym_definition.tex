%% Acronym definition example using glossaries package
%% \usepackage{acro} is required
%% 
%% For a powerful usage of the acro package look at http://tex.stackexchange.com/questions/135975/how-to-define-an-acronym-by-using-other-acronym-and-print-the-abbreviations-toge

\DeclareAcronym{cnn}{
  short = CNN,
  long = Condensed Nearest Neighbour
}

\DeclareAcronym{nn}{
  short = NN,
  long = Nearest Neighbour
}

\DeclareAcronym{oss}{
  short = OSS,
  long = One-Sided Selection
}

\DeclareAcronym{smote}{
  short = SMOTE,
  long = Synthetic Minority Over-sampling TEchnique
}

\DeclareAcronym{enn}{
  short = ENN,
  long = Edited Nearest Neighbour
}

\DeclareAcronym{svm}{
  short = SVM,
  long = Support Vector Machines
}

\DeclareAcronym{cad}{
  short = CAD, 
  long = Computer-Aided Diagnosis
}

\DeclareAcronym{clbp}{
  short = CLBP,
  long = Completed Local Binary Pattern
}
\DeclareAcronym{lbp}{
  short = LBP,
  long = Local Binary Pattern
}
\DeclareAcronym{rf}{
  short = RF,
  long = Random Forests
}

\DeclareAcronym{ncr}{
  short = NCR,
  long = Neighborhood Cleaning Rule
}

\DeclareAcronym{se}{
  short = SE,
  long = Sensitivity
}

\DeclareAcronym{sp}{
  short = SP,
  long =  Specificity
}

\DeclareAcronym{wracc}{
  short = WRacc,
  long = Weighted Relative Accuracy
}

\DeclareAcronym{fpr}{
  short = FPR,
  long = False Positive Rate
}

\DeclareAcronym{nm}{
  short = NM,
  long = NearMiss
}

\DeclareAcronym{nm1}{
  short = NM1,
  long = NearMiss-1
}

\DeclareAcronym{nm2}{
  short = NM2,
  long = NearMiss-2
}

\DeclareAcronym{nm3}{
  short = NM3,
  long = NearMiss-3
}

\DeclareAcronym{os}{
  short = OS,
  long = Over-Sampling
}

\DeclareAcronym{dos}{
  short = DOS,
  long = Data Over-Sampling
}
\DeclareAcronym{ros}{
  short = ROS,
  long = Random Over-Sampling
}

\DeclareAcronym{us}{
  short = US,
  long = Under-Sampling
}

\DeclareAcronym{rus}{
  short = RUS,
  long = Random Under-Sampling
}

\DeclareAcronym{cus}{
  short = CUS,
  long = Clustering Under-Sampling
}

\DeclareAcronym{bd}{
  short = BD,
  long = Barrel Deformation
}

\DeclareAcronym{rdgm}{
  short = RDGM,
  long = Random Deformation using Gaussian Motion 
}

\DeclareAcronym{tl}{
  short = TL,
  long = Tomek Link 
}

\DeclareAcronym{auc}{
  short = AUC,
  long = Area Under the Curve
}